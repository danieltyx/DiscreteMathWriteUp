\documentclass[12pt]{article}
\usepackage{amsmath}
\usepackage{amssymb,amsfonts,latexsym,pgfplots,polynom,mathpazo,enumitem,textcase,bm,amsthm,fancyhdr}
\usepackage[utf8]{inputenc}
\usepackage[english]{babel}
\usepackage[margin=.9in, tmargin=1.5in, bmargin=1in]{geometry}
\usepackage{physics}
\usepackage{diffcoeff}

%pgfplots stuff
\pgfplotsset{width=10cm,compat=1.9}
\usepgfplotslibrary{external}
%\tikzexternalize 


%Theorems and stuff
\newtheorem{theorem}{Theorem}[section]
\newtheorem{corollary}{Corollary}[theorem]
\newtheorem{lemma}[theorem]{Lemma}
\newtheorem*{remark}{Remark}
\newtheorem{definition}{Definition}[section]

%Sets and stuff
\newcommand{\Z}{\mathbb{Z}}
\newcommand{\R}{\mathbb{R}}
\newcommand{\Q}{\mathbb{Q}}
\newcommand{\N}{\mathbb{N}}
\newcommand{\J}{\mathbb{J}}
\newcommand{\C}{\mathbb{C}}

\renewcommand\qedsymbol{$\blacksquare$}
\newenvironment{solution}
{\begin{proof}[Solution]\renewcommand\qedsymbol{$\square$}}
	{\end{proof}}

%Sequences and basic analysis
\newcommand{\seq}[1]{\{{#1}_n\}_{n=1}^\infty}
\newcommand{\seqk}[1]{\{{#1}_k\}_{k=1}^\infty}
\newcommand{\sseq}[1]{\{{#1}_{n_k}\}_{k=1}^\infty}
\newcommand{\script}[1]{\mathcal{#1}}
\newcommand{\Lim}[1]{\lim\limits_{{#1}\rightarrow\infty}}
\newcommand{\Limsup}[1]{\overline{\lim\limits_{{#1}\rightarrow\infty}}\textrm{ }}
\newcommand{\Liminf}[1]{\underline{\lim\limits_{{#1}\rightarrow\infty}}}
\newcommand{\re}{\textrm{Re}}
\newcommand{\im}{\textrm{Im}}

%making things bigger 
\newcommand{\Frac}[2]{\displaystyle\frac{#1}{#2}}
\newcommand{\Int}[2]{\displaystyle\int_{#1}^{#2}}
\newcommand{\Sum}[2]{\displaystyle\sum_{#1}^{#2}}
\newcommand{\Heq}{\overset{\mathrm{H}}{=}}
\newcommand{\dist}{\textrm{dist}}
\newcommand{\rpm}{\sbox0{$1$}\sbox2{$\scriptstyle\pm$}
	\raise\dimexpr(\ht0-\ht2)/2\relax\box2 }

%formatting
\newcommand\textlcsc[1]{\textsc{\MakeTextLowercase{#1}}}
\newcommand{\tab}{\hspace{10mm}}

%Vectors
\newcommand{\X}{\textbf{X}}
\newcommand{\Y}{\textbf{Y}}
\newcommand{\U}{\textbf{U}}
\newcommand{\vi}{\textbf{i}}
\newcommand{\vj}{\textbf{j}}
\newcommand{\vk}{\textbf{k}}
\newcommand{\vr}{\textbf{r}}
\newcommand{\vv}{\textbf{v}}
\newcommand{\vcu}{\textbf{u}}
\newcommand{\vca}{\textbf{a}}
\newcommand{\vcb}{\textbf{b}}
\newcommand{\vc}{\textbf{c}}
\newcommand{\la}{\langle}
\newcommand{\ra}{\rangle}

\newcommand{\pfpu}{\dfrac{\partial f}{\partial  u}}
\newcommand{\pfpv}{\dfrac{\partial f}{\partial  v}}
\newcommand{\pfpw}{\dfrac{\partial f}{\partial  w}}
\newcommand{\pfpx}{\dfrac{\partial f}{\partial  x}}
\newcommand{\pfpy}{\dfrac{\partial f}{\partial  y}}
\newcommand{\pupx}{\dfrac{\partial u}{\partial  x}}
\newcommand{\pupy}{\dfrac{\partial u}{\partial  y}}
\newcommand{\pupz}{\dfrac{\partial u}{\partial  z}}
\newcommand{\pvpx}{\dfrac{\partial v}{\partial  x}}
\newcommand{\pvpy}{\dfrac{\partial v}{\partial  y}}
\newcommand{\pvpz}{\dfrac{\partial v}{\partial  z}}
\newcommand{\pwpx}{\dfrac{\partial w}{\partial  x}}
\newcommand{\pwpy}{\dfrac{\partial w}{\partial  y}}
\newcommand{\pwpz}{\dfrac{\partial w}{\partial  z}}
\renewcommand\arraystretch{1.2}
%integrals
\def\upint{\mathchoice%
	{\mkern13mu\overline{\vphantom{\intop}\mkern7mu}\mkern-20mu}%
	{\mkern7mu\overline{\vphantom{\intop}\mkern7mu}\mkern-14mu}%
	{\mkern7mu\overline{\vphantom{\intop}\mkern7mu}\mkern-14mu}%
	{\mkern7mu\overline{\vphantom{\intop}\mkern7mu}\mkern-14mu}%
	\int}
\def\lowint{\mkern3mu\underline{\vphantom{\intop}\mkern7mu}\mkern-10mu\int}

%partials
\newcommand{\partd}[2]{\frac{\partial {#1}}{\partial {#2}}}
\newcommand{\partdd}[2]{\frac{\partial^2 {#1}}{\partial {#2}^2}}
\newcommand{\Partdd}[3]{\frac{\partial^2 {#1}}{\partial {#2}\partial{#3}}}
\newcommand{\Partddd}[4]{\frac{\partial^3 {#1}}{\partial {#2}\partial{#3}\partial{#4}}}
\newcommand{\Partdddd}[5]{\frac{\partial^4 {#1}}{\partial {#2}\partial{#3}\partial{#4}\partial{#5}}}

%words in math commands
\newcommand{\mathand}{\quad\textrm{and}\quad }
\newcommand{\st}{\textrm{ such that }}
\newcommand{\as}{\textrm{ as }}
\newcommand{\fs}{\textrm{ for some }}

%matrices
\newenvironment{amatrix}[1]{%
	\left[\begin{array}{@{}*{#1}{c}|c@{}}
	}{%
	\end{array}\right]
}



\pagestyle{fancy}
\fancyhf{}
\rhead{Daniel Tian\\ Advanced Topics in Math, Foil}   
\lhead{9/19/22}
\chead{\bf \large Write Up 1}
\cfoot{Page \thepage}


\begin{document}
\begin{enumerate}
	\item 
	Prove that an integer is odd if and only if it is the sum of two consecutive integers.
	\begin{proof}
		Let $x$ be the integer\\
		($\Rightarrow$) Suppose $x$ is odd, so there is an integer $a$ such that $x = 2a+1$. $x=2a+1=a+a+1=a+(a+1)$, and since $a$ is integer, $a+1$ must also be a integer. Hence, $x$ is the sum of two consecutive integers $a$ and $a+1$.\\
		($\Leftarrow$) Suppose $x$ is the sum of two consecutive integers. Let the two consecutive integer be $a$ and $a+1$. Then, we have $x=a+(a+1)=2a+1$. By defination, $x$ is an odd number.
	\end{proof}
	\item 
	Let $x$ and $y$ be integers. Prove that $x \leq y-1$ if and only if $x<y$.
	\begin{proof}
		Let $x$ and $y$ be integers. \\
		($\Rightarrow$) Suppose $x \leq y-1$. It indicates $y-x \geq 1$ after organize it according to the property of inequalities. Since  $1>0$, we have 	$y-x \geq 1 > 0 $ and hence $y-x >0$. Move $x$ to the opposite side, we can conclude that $x<y$.\\
		($\Leftarrow$) Suppose $x<y$. This indicates that $y-x>0$. We know that $y-x$ is integer since both $y$ and $x$ are integers.	Sinice there difference is greater than zero and is a integer, it needs to be at least 1. Hence, we have $y-x\geq 1$
		\begin{align*}
			y-x &\geq 1\\
			1 &\leq y-x \\
			x+1 &\leq y\\
			x &\leq y-1\\
		\end{align*}
		By organizing the inequality, we can conclude that $x \leq y-1$.
	\end{proof}

\item Let $a$ be an integer. Prove that $a$ is odd if and only if there is an integer $x$ such that $a = 2x - 1$.	\begin{proof}
		Let $a$ be an integer. \\
		($\Rightarrow$) Suppose $a$ is odd. By defination, there exist an integer $b$ such that $a=2b+1$.
		Let integer $x=b+1$. We can rewrite $a=2b+1 = 2(b+1) -1 = 2x-1$.
		Hence, we can conclude $a=2x-1$.  \\
	   ($\Leftarrow$) Suppose there is an integer $x$ such that $a=2x-1$. We can rewrite it as $a=2(x-1)+1$.
	   Let integer $y=x-1$. Then, we will have $a=2y+1$, and a is odd clearly.
\end{proof}

	\item Prove that the difference between consecutive perfect squares is odd.
	\begin{proof}
		Let x be an integer. Then $x^2$ and $(x+1)^2$ must be consecutive perfect square. Let $a$ be there difference and $a = (x+1)^2-x^2$. 
		\begin{align*}
			&= (x+1)^2-x^2 \\
			&=  x^2 + 2x+1-x^2\\
			&= 2x+1\\
		\end{align*}
		Therefore, $a=2x+1$, by defination, $a$ is odd and hence the difference is odd. 
	\end{proof}
	\item Disprove the proposition: Two right triangles have the same area is and only if the lengths of their hypotenuses are the same.
	\begin{proof}
		Let the first triangle has legs of 4 and 6 and the second triangle has legs of 3 and 8. They have the same area $4*6*\frac{1}{2} = 3*8*\frac{1}{2} = 12$. However, the hypotenuse of the first triangle is $\sqrt{4^2+6^2} =\sqrt{52}$ and the hypotenuse of the second triangle is $\sqrt{3^2+8^2} =\sqrt{73}.$ Obviously, $\sqrt{52} \ne \sqrt{73}$ according the Pythagorean theorem . Hence, the length of their hypotenuses are not the same, the statement is disproved. 
	\end{proof}

\end{enumerate}

\end{document}