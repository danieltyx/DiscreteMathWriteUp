\documentclass[12pt]{article}
\usepackage{amsmath}
\usepackage{amssymb,amsfonts,latexsym,pgfplots,polynom,mathpazo,enumitem,textcase,bm,amsthm,fancyhdr}
\usepackage[utf8]{inputenc}
\usepackage[english]{babel}
\usepackage[margin=.9in, tmargin=1.5in, bmargin=1in]{geometry}
\usepackage{physics}
\usepackage{diffcoeff}
\usepackage{listings}


%pgfplots stuff
\pgfplotsset{width=10cm,compat=1.9}
\usepgfplotslibrary{external}
%\tikzexternalize 


%Theorems and stuff
\newtheorem{theorem}{Theorem}[section]
\newtheorem{corollary}{Corollary}[theorem]
\newtheorem{lemma}[theorem]{Lemma}
\newtheorem*{remark}{Remark}
\newtheorem{definition}{Definition}[section]

%Sets and stuff
\newcommand{\Z}{\mathbb{Z}}
\newcommand{\R}{\mathbb{R}}
\newcommand{\Q}{\mathbb{Q}}
\newcommand{\N}{\mathbb{N}}
\newcommand{\J}{\mathbb{J}}
\newcommand{\C}{\mathbb{C}}

\renewcommand\qedsymbol{$\blacksquare$}
\newenvironment{solution}
{\begin{proof}[Solution]\renewcommand\qedsymbol{$\square$}}
	{\end{proof}}

%Sequences and basic analysis
\newcommand{\seq}[1]{\{{#1}_n\}_{n=1}^\infty}
\newcommand{\seqk}[1]{\{{#1}_k\}_{k=1}^\infty}
\newcommand{\sseq}[1]{\{{#1}_{n_k}\}_{k=1}^\infty}
\newcommand{\script}[1]{\mathcal{#1}}
\newcommand{\Lim}[1]{\lim\limits_{{#1}\rightarrow\infty}}
\newcommand{\Limsup}[1]{\overline{\lim\limits_{{#1}\rightarrow\infty}}\textrm{ }}
\newcommand{\Liminf}[1]{\underline{\lim\limits_{{#1}\rightarrow\infty}}}
\newcommand{\re}{\textrm{Re}}
\newcommand{\im}{\textrm{Im}}

%making things bigger 
\newcommand{\Frac}[2]{\displaystyle\frac{#1}{#2}}
\newcommand{\Int}[2]{\displaystyle\int_{#1}^{#2}}
\newcommand{\Sum}[2]{\displaystyle\sum_{#1}^{#2}}
\newcommand{\Heq}{\overset{\mathrm{H}}{=}}
\newcommand{\dist}{\textrm{dist}}
\newcommand{\rpm}{\sbox0{$1$}\sbox2{$\scriptstyle\pm$}
	\raise\dimexpr(\ht0-\ht2)/2\relax\box2 }

%formatting
\newcommand\textlcsc[1]{\textsc{\MakeTextLowercase{#1}}}
\newcommand{\tab}{\hspace{10mm}}

%Vectors
\newcommand{\X}{\textbf{X}}
\newcommand{\Y}{\textbf{Y}}
\newcommand{\U}{\textbf{U}}
\newcommand{\vi}{\textbf{i}}
\newcommand{\vj}{\textbf{j}}
\newcommand{\vk}{\textbf{k}}
\newcommand{\vr}{\textbf{r}}
\newcommand{\vv}{\textbf{v}}
\newcommand{\vcu}{\textbf{u}}
\newcommand{\vca}{\textbf{a}}
\newcommand{\vcb}{\textbf{b}}
\newcommand{\vc}{\textbf{c}}
\newcommand{\la}{\langle}
\newcommand{\ra}{\rangle}

\newcommand{\pfpu}{\dfrac{\partial f}{\partial  u}}
\newcommand{\pfpv}{\dfrac{\partial f}{\partial  v}}
\newcommand{\pfpw}{\dfrac{\partial f}{\partial  w}}
\newcommand{\pfpx}{\dfrac{\partial f}{\partial  x}}
\newcommand{\pfpy}{\dfrac{\partial f}{\partial  y}}
\newcommand{\pupx}{\dfrac{\partial u}{\partial  x}}
\newcommand{\pupy}{\dfrac{\partial u}{\partial  y}}
\newcommand{\pupz}{\dfrac{\partial u}{\partial  z}}
\newcommand{\pvpx}{\dfrac{\partial v}{\partial  x}}
\newcommand{\pvpy}{\dfrac{\partial v}{\partial  y}}
\newcommand{\pvpz}{\dfrac{\partial v}{\partial  z}}
\newcommand{\pwpx}{\dfrac{\partial w}{\partial  x}}
\newcommand{\pwpy}{\dfrac{\partial w}{\partial  y}}
\newcommand{\pwpz}{\dfrac{\partial w}{\partial  z}}
\renewcommand\arraystretch{1.2}
%integrals
\def\upint{\mathchoice%
	{\mkern13mu\overline{\vphantom{\intop}\mkern7mu}\mkern-20mu}%
	{\mkern7mu\overline{\vphantom{\intop}\mkern7mu}\mkern-14mu}%
	{\mkern7mu\overline{\vphantom{\intop}\mkern7mu}\mkern-14mu}%
	{\mkern7mu\overline{\vphantom{\intop}\mkern7mu}\mkern-14mu}%
	\int}
\def\lowint{\mkern3mu\underline{\vphantom{\intop}\mkern7mu}\mkern-10mu\int}

%partials
\newcommand{\partd}[2]{\frac{\partial {#1}}{\partial {#2}}}
\newcommand{\partdd}[2]{\frac{\partial^2 {#1}}{\partial {#2}^2}}
\newcommand{\Partdd}[3]{\frac{\partial^2 {#1}}{\partial {#2}\partial{#3}}}
\newcommand{\Partddd}[4]{\frac{\partial^3 {#1}}{\partial {#2}\partial{#3}\partial{#4}}}
\newcommand{\Partdddd}[5]{\frac{\partial^4 {#1}}{\partial {#2}\partial{#3}\partial{#4}\partial{#5}}}

%words in math commands
\newcommand{\mathand}{\quad\textrm{and}\quad }
\newcommand{\st}{\textrm{ such that }}
\newcommand{\as}{\textrm{ as }}
\newcommand{\fs}{\textrm{ for some }}

%matrices
\newenvironment{amatrix}[1]{%
	\left[\begin{array}{@{}*{#1}{c}|c@{}}
	}{%
	\end{array}\right]
}



\pagestyle{fancy}
\fancyhf{}
\rhead{Daniel Tian\\ Advanced Topics in Math, Foil}   
\lhead{10/3/22}
\chead{\bf \large Write Up 3}
\cfoot{Page \thepage}


\begin{document}
	\begin{enumerate}
		\item 
		Let $n$ be a positive integer. Prove that $n^2 = (n)_2 + n$ in the following two ways.
		
		\begin{enumerate}[label=(\alph*)]
			\item 
			First, show it is true algebraically.
			\begin{solution}
				\begin{align*}
					n^2 &= (n)_2 + n\\
					&= n(n-1) + n\\
					&= n^2 - n + n\\
					&= n^2
				\end{align*}
			\end{solution}
			\item
			Second, interpret the terms $n^2$ and $(n)_2$, and $n$ in the context of list counting and then argue that the equation must be true.
			\begin{solution}
				$n^2$ refers to the number of ways a 2-element list can be formed out of $n$ digits, where repetition is allowed.\\
				$(n)_2$ is the number of ways a 2-element list can be formed out of $n$ digits without repetition, and $n$ is the number of repeated 2-element lists. Therefore, by adding $(n)_2$ and $n$ together, we get the total number of ways a 2-element list can be formed out of $n$ digits regardless of repetition, which is the same concept as $n^2$.
			\end{solution}
			
		\end{enumerate}
		
		\item
		Prove that all of the following numbers are composite.
		\[1000! + 2,  1000! + 3,  1000! + 4, . . . ,  1000! + 1002.\]
		\begin{proof}
			We can rewrite the numbers in the proposition into $(1000!+n)$, where $n\in \Z$ and $2\leq n\leq 1002$. Thus, we conclude that $n|(1000!+n)$, when $2\leq n\leq 1000$\\
			Notice that when $2\leq n\leq 1000$, $n$ is always a factor of $(1000!)$ by the definition of factorial. Thus, we can rewrite the expression into \[ n\times (\frac{1000!}{n}), \] where it is guaranteed $\displaystyle \frac{1000!}{n}\in \Z$.\\
			When $n=1001$, we find that $7\times 143 = 1001$. Since $2\leq 7\leq 1000$, implying 7 is also a factor of $1000!$, we get that $\displaystyle 1000!+1001 = 1000!+7\times 143 = 7\times \frac{1000!}{7}$. Thus, $7|(1000!+1001)$.\clearpage
			Likewise, since $2\times 501 = 1002$, we conclude that $2|(1000!+1002)$.\\ 
			In conclusion, there always exists an integer $a$ such that $a|(1000!+n)$, where $2\leq n\leq 1002$, and $1<a\leq (1000!+n)$, which is the definition of the composite number.
		\end{proof}
		
		\item
		The double factorial $n!!$ is definted for odd positive integers $n$; it is the product of all the odd numbers from 1 to $n$ inclusive. For example, $7!! = 7 \times 5 \times 3 \times 1 = 105$. Answer
		the following:
		\begin{enumerate}[label=(\alph*)]
			\item 
			Evaluate $9!!$.
			\begin{solution}
				\[ 9!! = 9\times 7\times 5\times 3\times 1 = 945 \]
			\end{solution}
			
			\item
			For an odd integer $n$, are $n!!$ and $(n!)!$ equal?
			\begin{solution}
				\begin{align*}
					n!!   &= n(n-2)(n-4)(n-6)...(n-n+1)\\
					n!    &= n(n-1)(n-2)(n-3)...(n-n+1)\\
					(n!)! &= n!(n!-1)(n!-2)(n!-3)...(n!-n!+1)\\
				\end{align*}
				Clearly, $\displaystyle n!!\neq (n!)!$. So they are not equal.
			\end{solution}
			
			\item
			Write an expression for $n!!$ using product notation.
			\begin{solution}
				\[ n!! = n(n-2)(n-4)(n-6)...(n-n+1) \]
			\end{solution}
			
			\item
			Explain why the following formula works:
			\[ (2k-1)!! = \frac{(2k)!}{k!2^k} \].
			
			
		\end{enumerate}
		
		\item
		Evaluate the following integral for $n = 0, 1, 2, 3, 4$:
		\[ \int_{0}^{\infty} x^ne^{-x} \,dx \]
		(Note: The case $n = 0$ is easiest. Do the remaining values of $n$ in order, and use integration by parts.) What is the value of this integral for an arbitrary natural number $n$? Using a calculation device, evaluate the integral for $n = \frac{1}{2}$. What is surprising about your ability to compute this, given your conclusion to the previous question?
		
		\item
		Let $A$, $B$, and $C$ be sets, and suppose $A \subseteq B$, $B \subseteq C$, and $C \subseteq A$. Prove that $A = C$.
		
	\end{enumerate}
\end{document}