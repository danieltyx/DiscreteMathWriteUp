\documentclass[12pt]{article}
\usepackage{amsmath}
\usepackage{amssymb,amsfonts,latexsym,pgfplots,polynom,mathpazo,enumitem,textcase,bm,amsthm,fancyhdr}
\usepackage[utf8]{inputenc}
\usepackage[english]{babel}
\usepackage[margin=.9in, tmargin=1.5in, bmargin=1in]{geometry}
\usepackage{physics}
\usepackage{diffcoeff}
\usepackage{listings}
\usepackage{amsfonts} 

%pgfplots stuff
\pgfplotsset{width=10cm,compat=1.9}
\usepgfplotslibrary{external}
%\tikzexternalize 


%Theorems and stuff
\newtheorem{theorem}{Theorem}[section]
\newtheorem{corollary}{Corollary}[theorem]
\newtheorem{lemma}[theorem]{Lemma}
\newtheorem*{remark}{Remark}
\newtheorem{definition}{Definition}[section]

%Sets and stuff
\newcommand{\Z}{\mathbb{Z}}
\newcommand{\R}{\mathbb{R}}
\newcommand{\Q}{\mathbb{Q}}
\newcommand{\N}{\mathbb{N}}
\newcommand{\J}{\mathbb{J}}
\newcommand{\C}{\mathbb{C}}

\renewcommand\qedsymbol{$\blacksquare$}
\newenvironment{solution}
{\begin{proof}[Solution]\renewcommand\qedsymbol{$\square$}}
	{\end{proof}}

%Sequences and basic analysis
\newcommand{\seq}[1]{\{{#1}_n\}_{n=1}^\infty}
\newcommand{\seqk}[1]{\{{#1}_k\}_{k=1}^\infty}
\newcommand{\sseq}[1]{\{{#1}_{n_k}\}_{k=1}^\infty}
\newcommand{\script}[1]{\mathcal{#1}}
\newcommand{\Lim}[1]{\lim\limits_{{#1}\rightarrow\infty}}
\newcommand{\Limsup}[1]{\overline{\lim\limits_{{#1}\rightarrow\infty}}\textrm{ }}
\newcommand{\Liminf}[1]{\underline{\lim\limits_{{#1}\rightarrow\infty}}}
\newcommand{\re}{\textrm{Re}}
\newcommand{\im}{\textrm{Im}}

%making things bigger 
\newcommand{\Frac}[2]{\displaystyle\frac{#1}{#2}}
\newcommand{\Int}[2]{\displaystyle\int_{#1}^{#2}}
\newcommand{\Sum}[2]{\displaystyle\sum_{#1}^{#2}}
\newcommand{\Heq}{\overset{\mathrm{H}}{=}}
\newcommand{\dist}{\textrm{dist}}
\newcommand{\rpm}{\sbox0{$1$}\sbox2{$\scriptstyle\pm$}
	\raise\dimexpr(\ht0-\ht2)/2\relax\box2 }

%formatting
\newcommand\textlcsc[1]{\textsc{\MakeTextLowercase{#1}}}
\newcommand{\tab}{\hspace{10mm}}

%Vectors
\newcommand{\X}{\textbf{X}}
\newcommand{\Y}{\textbf{Y}}
\newcommand{\U}{\textbf{U}}
\newcommand{\vi}{\textbf{i}}
\newcommand{\vj}{\textbf{j}}
\newcommand{\vk}{\textbf{k}}
\newcommand{\vr}{\textbf{r}}
\newcommand{\vv}{\textbf{v}}
\newcommand{\vcu}{\textbf{u}}
\newcommand{\vca}{\textbf{a}}
\newcommand{\vcb}{\textbf{b}}
\newcommand{\vc}{\textbf{c}}
\newcommand{\la}{\langle}
\newcommand{\ra}{\rangle}

\newcommand{\pfpu}{\dfrac{\partial f}{\partial  u}}
\newcommand{\pfpv}{\dfrac{\partial f}{\partial  v}}
\newcommand{\pfpw}{\dfrac{\partial f}{\partial  w}}
\newcommand{\pfpx}{\dfrac{\partial f}{\partial  x}}
\newcommand{\pfpy}{\dfrac{\partial f}{\partial  y}}
\newcommand{\pupx}{\dfrac{\partial u}{\partial  x}}
\newcommand{\pupy}{\dfrac{\partial u}{\partial  y}}
\newcommand{\pupz}{\dfrac{\partial u}{\partial  z}}
\newcommand{\pvpx}{\dfrac{\partial v}{\partial  x}}
\newcommand{\pvpy}{\dfrac{\partial v}{\partial  y}}
\newcommand{\pvpz}{\dfrac{\partial v}{\partial  z}}
\newcommand{\pwpx}{\dfrac{\partial w}{\partial  x}}
\newcommand{\pwpy}{\dfrac{\partial w}{\partial  y}}
\newcommand{\pwpz}{\dfrac{\partial w}{\partial  z}}
\renewcommand\arraystretch{1.2}
%integrals
\def\upint{\mathchoice%
	{\mkern13mu\overline{\vphantom{\intop}\mkern7mu}\mkern-20mu}%
	{\mkern7mu\overline{\vphantom{\intop}\mkern7mu}\mkern-14mu}%
	{\mkern7mu\overline{\vphantom{\intop}\mkern7mu}\mkern-14mu}%
	{\mkern7mu\overline{\vphantom{\intop}\mkern7mu}\mkern-14mu}%
	\int}
\def\lowint{\mkern3mu\underline{\vphantom{\intop}\mkern7mu}\mkern-10mu\int}

%partials
\newcommand{\partd}[2]{\frac{\partial {#1}}{\partial {#2}}}
\newcommand{\partdd}[2]{\frac{\partial^2 {#1}}{\partial {#2}^2}}
\newcommand{\Partdd}[3]{\frac{\partial^2 {#1}}{\partial {#2}\partial{#3}}}
\newcommand{\Partddd}[4]{\frac{\partial^3 {#1}}{\partial {#2}\partial{#3}\partial{#4}}}
\newcommand{\Partdddd}[5]{\frac{\partial^4 {#1}}{\partial {#2}\partial{#3}\partial{#4}\partial{#5}}}

%words in math commands
\newcommand{\mathand}{\quad\textrm{and}\quad }
\newcommand{\st}{\textrm{ such that }}
\newcommand{\as}{\textrm{ as }}
\newcommand{\fs}{\textrm{ for some }}

%matrices
\newenvironment{amatrix}[1]{%
	\left[\begin{array}{@{}*{#1}{c}|c@{}}
	}{%
	\end{array}\right]
}



\pagestyle{fancy}
\fancyhf{}
\rhead{Daniel Tian\\ Advanced Topics in Math, Foil}   
\lhead{2/15/23}
\chead{\bf \large Write Up 7}
\cfoot{Page \thepage}
\begin{document}

\textbf{For problems 1-3, let $H, K \leq G$.}
 \begin{enumerate}
    \item Prove that $H \subseteq K \implies H \leq K$.
    
    \begin{proof}
        

	Let $H \subseteq K$. Since $K$ is a group and $H$ is a nonempty 
    subset of $K$. We are sure it is nonempty because $H$ is a subgroup of $G$,
    where it cannot be empty.
    If we can show $H$ is closed under the group operation  
     and inverses, we can say that $H$ is a subgroup 
    of $K$.
    
(1) Let $h_1$ and $h_2$ be elements of $H$. Since $H \leq G$, $H$ is closed with the group operation and $h_1*h_2$ is in $H$. 
Since $H,K \subseteq G$, they all share the same operation.
%Then, $H$ is a subgroup of $K$, which means every elements in $H$ is also in $K$. Hence, $h_1*h_2$ is in $K$.
 Therefore, $H$ is closed under
the group operaton.

(2) For each element $h$ in $H$, its inverse is also in $H$ because $H$ is a group. Therefore, $H$ has inverses for all its elements.

We can conclude that $H$ is a subgroup of $K$, H $\leq $K.
\end{proof}

    \item Show that $H \cap K \leq G$.
    \begin{proof}
        Let $H \cap K \leq G$. Since $G$ is a group and $H \cap K $ is a nonempty 
        subset of $G$. We are sure the set is nonempty because they are both subgroups of G and hence
        have to share at least one element, the identity element.
        Then, if we can show $H \cap K$ is closed under the group operation  
         and inverses, we can say that $H \cap K$ is a subgroup of $G$.

        (1) We know that for every element $x$ in $H \cap K$, $x \in H$ and 
        $x \in K$. Let $x_1$ and $x_2$ be elements of $H \cap K$. $x_1 * x_2$ must be in $H$ as $H$ 
        is a group closed under the operation and $x1$ and $x2$ are both in $H$. Similarly, $x_1 * x_2$ 
        must be in $K$. Hence, $x_1 * x_2$ is in $H \cap K$, it is closed under the operation.

        (2) Let $x'$ be the inverse of $x$ in $H \cap K$. Similarly, since $x$ in both in $H$ and $K$, 
        its inverse is in both $H$ and $K$. Hence, $H \cap K$ has inverses for all its elements.


    \end{proof}
 
    \item Let $G$ be an abelian group, and define $HK$ as follows:
        \[HK = \{hk \mid h \in H \text{  and  } k \in K\}\]
        Prove that $HK \leq G$.
        \begin{proof}
            To prove that $HK \leq G$, we need to show that (1) $HK$ is a subset of $G$.
            (2) $HK$ is closed under the opreation and (3) inverses.\\

            (1) Since $H,K \subseteq G$, all elements in $H$ and $K$ are in $G$. Also that G is a group,
            so it is closed under the group opreation, hence, for every $h \in H$ and $k \in K$, $hk$ must also 
            be in $G$. Hence, $HK$ is a subset of $G$.
            
            (2)Let $h_1, h_2 \in H$ and $k_1,k_2 \in K$. $(h_1k_1)(h_2k_2) = (h_1h_2)(k_1k_2)$ because
            $G$ is abelian group and $H,K$ are also abelian groups. It implies the communitivity. We know that
            $h_1h_2 \in H$ and $k_1k_2 \in K$ because $H$ and $K$ are groups. Therefore, $HK$ is closed under 
            the group operation.

            (3) Let $hk \in H$, and $(hk)^{-1}$ be its inverse. Since $hk$ is in $G$ its inverse $(hk)^{-1}$ must
            also be in $G$. Hence, $(hk)^{-1} = k^{-1} h^{-1} = h^{-1} k^{-1} $, by the communitivity from the abelian group.
            Since $H$ and $K$ are both groups, $h^{-1} \in H$ and $k^{-1} \in K$. $(hk)^{-1} \in HK$.$HK$ has inverses for 
            all its elements.
        \end{proof}
 
    \item Suppose a group $G$ is generated by two elements $a$ and $b$.\\
        Prove that $ab = ba \implies G$ is abelian.
        \begin{proof}
			Let $ab=ba$. We need to prove that group $G$ has communitivity 
            for it to be a abelian group. \\
            Since $ab=ba$, we can conclude that $a$ and $b$ commute. We can also prove $a$ and $b^{-1}$ commute,
            $ab^{-1} = b^{-1}bab^{-1} = b^{-1}abb^{-1} = b^{-1}$; $a^{-1}$ and $b$ commute, $a^{-1}b = a^{-1}baa^{-1}
            = a^{-1}aba^{-1} = ba^{-1}$; $a^{-1}$ and $b^{-1}$ commute, $ a^{-1}b^{-1} = a^{-1}b^{-1}aa^{-1} = a^{-1}ab^{-1}a^{-1}
            =b^{-1}a^{-1}$.\\
            Since the group $G$ is generated by $a$ and $b$, every element can be written out
            only using $a$, $a^{-1}$, $b$ and $b^{-1}$. We proved that $a$ and $b^{-1}$, $b$ and $a^{-1}$,
            and $a^{-1}$ and $b^{-1}$ commute. Given that $a$ and $b$ commute, and inherently $a$ commute with $a$ and $b$ commute
            with $b$. We can conclude that all elements commute with each other in any generated elements by $a$ and $b$. Therefore, 
            we will be able to simplify any element in the group $G$ as $a^xb^y$ (x,y $\in \Z$) after moving all $a$ terms to the left side by communitivity,
            and combining like items. Similarly, we can moving all $b$ terms to the left side and reduce it to $b^ya^x$. It implies that $a^xb^y$  
            and $b^ya^x$ are identical. Hence $a^x$ commute with $b^y$.
            \\
            Now, we need to prove that $a^xb^y$ commute with $a^qb^p$.\\
             \begin{align*}
                (a^xb^y)(a^qb^p) &= a^x(b^ya^q)b^p  \tab \tab &\text{associative}\\
                &= a^x(a^qb^y)b^p  \tab \tab &\text{$a^q$ commute with $b^y$}\\
                &= (a^xa^q)(b^yb^p) \tab \tab &\text{associative}\\
                &= a^{x+q}b^{y+p}\\
                & = a^qa^xb^pb^y\\
                & = a^qb^pa^xb^y \tab \tab &\text{associative} \\
                & = (a^qb^p)(a^xb^y) 
                \\
             \end{align*}
            Therefore, we proved that group G has communitivity and hence it is abelian.
        \end{proof}

    \item Define the center of a group to be
        \[C = \{g \in G \mid gx = xg, \forall x \in G\},\]
        that is, the set of all elements of $G$ that commute with every element of $G$.\\
        Prove $C \leq G$.

        \begin{proof}
            We know that $C$ is a subset of the group $G$ because $ \forall g \in G$. Then, we need to prove
            (1) C is closed under the group operation and (2) inverses.

            (1) Let $c,d \in C$. We know that $cx = xc$ and $dx = xd$. Hence, $(cd)x=cdx = cxd = xcd= x(cd)$.
            Hence, $cd \in C$, C is closed under the group operation.\\
            (2) Let $c \in C$ and $c^{-1}$ be its inverse. $c^{-1}x=c^{-1}xe= c^{-1}xcc^{-1} = c^{-1}cxc^{-1} = exc^{-1} = xc^{-1}$. 
            Hence, $c^{-1} \in C$, C is closed under inverses.\\
            Therefore, $C$ is a subgroup of $G$, $C \leq G$.
        \end{proof}

	


 \end{enumerate}
\end{document}