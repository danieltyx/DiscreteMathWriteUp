\documentclass[12pt]{article}
\usepackage{amsmath}
\usepackage{amssymb,amsfonts,latexsym,pgfplots,polynom,mathpazo,enumitem,textcase,bm,amsthm,fancyhdr}
\usepackage[utf8]{inputenc}
\usepackage[english]{babel}
\usepackage[margin=.9in, tmargin=1.5in, bmargin=1in]{geometry}
\usepackage{physics}
\usepackage{diffcoeff}
\usepackage{listings}
\usepackage{amsfonts} 

%pgfplots stuff
\pgfplotsset{width=10cm,compat=1.9}
\usepgfplotslibrary{external}
%\tikzexternalize 


%Theorems and stuff
\newtheorem{theorem}{Theorem}[section]
\newtheorem{corollary}{Corollary}[theorem]
\newtheorem{lemma}[theorem]{Lemma}
\newtheorem*{remark}{Remark}
\newtheorem{definition}{Definition}[section]

%Sets and stuff
\newcommand{\Z}{\mathbb{Z}}
\newcommand{\R}{\mathbb{R}}
\newcommand{\Q}{\mathbb{Q}}
\newcommand{\N}{\mathbb{N}}
\newcommand{\J}{\mathbb{J}}
\newcommand{\C}{\mathbb{C}}

\renewcommand\qedsymbol{$\blacksquare$}
\newenvironment{solution}
{\begin{proof}[Solution]\renewcommand\qedsymbol{$\square$}}
	{\end{proof}}

%Sequences and basic analysis
\newcommand{\seq}[1]{\{{#1}_n\}_{n=1}^\infty}
\newcommand{\seqk}[1]{\{{#1}_k\}_{k=1}^\infty}
\newcommand{\sseq}[1]{\{{#1}_{n_k}\}_{k=1}^\infty}
\newcommand{\script}[1]{\mathcal{#1}}
\newcommand{\Lim}[1]{\lim\limits_{{#1}\rightarrow\infty}}
\newcommand{\Limsup}[1]{\overline{\lim\limits_{{#1}\rightarrow\infty}}\textrm{ }}
\newcommand{\Liminf}[1]{\underline{\lim\limits_{{#1}\rightarrow\infty}}}
\newcommand{\re}{\textrm{Re}}
\newcommand{\im}{\textrm{Im}}

%making things bigger 
\newcommand{\Frac}[2]{\displaystyle\frac{#1}{#2}}
\newcommand{\Int}[2]{\displaystyle\int_{#1}^{#2}}
\newcommand{\Sum}[2]{\displaystyle\sum_{#1}^{#2}}
\newcommand{\Heq}{\overset{\mathrm{H}}{=}}
\newcommand{\dist}{\textrm{dist}}
\newcommand{\rpm}{\sbox0{$1$}\sbox2{$\scriptstyle\pm$}
	\raise\dimexpr(\ht0-\ht2)/2\relax\box2 }

%formatting
\newcommand\textlcsc[1]{\textsc{\MakeTextLowercase{#1}}}
\newcommand{\tab}{\hspace{10mm}}

%Vectors
\newcommand{\X}{\textbf{X}}
\newcommand{\Y}{\textbf{Y}}
\newcommand{\U}{\textbf{U}}
\newcommand{\vi}{\textbf{i}}
\newcommand{\vj}{\textbf{j}}
\newcommand{\vk}{\textbf{k}}
\newcommand{\vr}{\textbf{r}}
\newcommand{\vv}{\textbf{v}}
\newcommand{\vcu}{\textbf{u}}
\newcommand{\vca}{\textbf{a}}
\newcommand{\vcb}{\textbf{b}}
\newcommand{\vc}{\textbf{c}}
\newcommand{\la}{\langle}
\newcommand{\ra}{\rangle}

\newcommand{\pfpu}{\dfrac{\partial f}{\partial  u}}
\newcommand{\pfpv}{\dfrac{\partial f}{\partial  v}}
\newcommand{\pfpw}{\dfrac{\partial f}{\partial  w}}
\newcommand{\pfpx}{\dfrac{\partial f}{\partial  x}}
\newcommand{\pfpy}{\dfrac{\partial f}{\partial  y}}
\newcommand{\pupx}{\dfrac{\partial u}{\partial  x}}
\newcommand{\pupy}{\dfrac{\partial u}{\partial  y}}
\newcommand{\pupz}{\dfrac{\partial u}{\partial  z}}
\newcommand{\pvpx}{\dfrac{\partial v}{\partial  x}}
\newcommand{\pvpy}{\dfrac{\partial v}{\partial  y}}
\newcommand{\pvpz}{\dfrac{\partial v}{\partial  z}}
\newcommand{\pwpx}{\dfrac{\partial w}{\partial  x}}
\newcommand{\pwpy}{\dfrac{\partial w}{\partial  y}}
\newcommand{\pwpz}{\dfrac{\partial w}{\partial  z}}
\renewcommand\arraystretch{1.2}
%integrals
\def\upint{\mathchoice%
	{\mkern13mu\overline{\vphantom{\intop}\mkern7mu}\mkern-20mu}%
	{\mkern7mu\overline{\vphantom{\intop}\mkern7mu}\mkern-14mu}%
	{\mkern7mu\overline{\vphantom{\intop}\mkern7mu}\mkern-14mu}%
	{\mkern7mu\overline{\vphantom{\intop}\mkern7mu}\mkern-14mu}%
	\int}
\def\lowint{\mkern3mu\underline{\vphantom{\intop}\mkern7mu}\mkern-10mu\int}

%partials
\newcommand{\partd}[2]{\frac{\partial {#1}}{\partial {#2}}}
\newcommand{\partdd}[2]{\frac{\partial^2 {#1}}{\partial {#2}^2}}
\newcommand{\Partdd}[3]{\frac{\partial^2 {#1}}{\partial {#2}\partial{#3}}}
\newcommand{\Partddd}[4]{\frac{\partial^3 {#1}}{\partial {#2}\partial{#3}\partial{#4}}}
\newcommand{\Partdddd}[5]{\frac{\partial^4 {#1}}{\partial {#2}\partial{#3}\partial{#4}\partial{#5}}}

%words in math commands
\newcommand{\mathand}{\quad\textrm{and}\quad }
\newcommand{\st}{\textrm{ such that }}
\newcommand{\as}{\textrm{ as }}
\newcommand{\fs}{\textrm{ for some }}

%matrices
\newenvironment{amatrix}[1]{%
	\left[\begin{array}{@{}*{#1}{c}|c@{}}
	}{%
	\end{array}\right]
}



\pagestyle{fancy}
\fancyhf{}
\rhead{Daniel Tian\\ Advanced Topics in Math, Foil}   
\lhead{4/6/23}
\chead{\bf \large Write Up 9}
\cfoot{Page \thepage}

\begin{document}
	If $G$ is a group, an \textbf{automorphism} of $G$ is an isomorphism from $G$ to $G$.
	\begin{enumerate}
		\item If $G$ is any group, and $a$ is any element of $G$, prove that $f(x) = axa^{-1}$ is an automorphism of $G$. We call this $conjugation$ by $b$.
		
	\begin{proof}
		We need to show that $f$ is a bijecton function and satisfy that $f(xy) = f(x)f(y)$ such that $x,y$ are both in $G$.
		Let $x,y \in G$ and $f(x) = f(y)$.
		\begin{align*}
			axa^{-1} &= aya^{-1} \\
			a^{-1}	axa^{-1} &= 	a^{-1}aya^{-1} \\
			xa^{-1} &= ya^{-1} \\
				xa^{-1}a &= ya^{-1}a \\
				x&=y
		\end{align*}
	Hence, $f$ is injective.
	Then, let us prove that for every $y \in G$ there exists $ x$ in G such that $f(x) = y$.
	 We have $x = a^{-1}ya$ satisfy the statement. $y = a( a^{-1}ya)a^{-1}= f(x)$ Hence, $f$ is surjective and hence bijective.
	 
	 Lastly, we need to show that $f(xy) = f(x)f(y)$ for all $x,y \in G$.
	 \begin{align*}
	 	f(xy) &= a(xy)a^{-1} \\
	 &= axya^{-1}\\
	 f(x)f(y) &= axa^{-1}  aya^{-1}\\
	  &= axeya^{-1}\\
	  	  &= axya^{-1}
	 \end{align*}
	 Hence, $f$ is isomorphism for all $x,y \in G$. Therefore, $f$ is an automorphism of $G$.
	 	\end{proof}
		\item Since each automorphism of $G$ is a bijective function from $G$ to $G$, it is a $permutation$ of $G$. Define Aut($G$) as the set of all automorphisms of $G$. Prove Aut($G$)$\leq S_G$.
		
		\begin{proof}
			First, we need to show that $Aut(G)$ is a nonempty subset of $S_G$. We can say that $\epsilon \in Aut(G)$ if $G$ is a isomorphism of $G$. Consider the identity function for $G$: $\epsilon(x) = x$. If $\epsilon(x) = \epsilon(y)$, then $x=y$. Hence the function is injective. For every $y$ in $G$, there exists an element $x \in G$ such that $\epsilon(y) = y$. Hence, the function is surjective and therefore bijective. Lastly, we know that  $\epsilon(xy) = xy = \epsilon(x)\epsilon(y)$. Therefore, $\epsilon$ is a isomorphism from $G$ to $G$ and hence $Aut(G)$ is a nonempty subset of $S_G$.
			
			Second, we need to show $Aut(G)$ is closed under composition. Let $f,g \in Aut(G)$, and we need to prove that the operation $f \circ g$ is in $Aut(G)$. That is, it is a automorphisms of $ G$, an isomorphism from G to G. Since both $f$ and $g$ are bijective, their composition will be bijective, which implies $f \circ g(x)  \in Aut(G) $ and $Aut(G)$ is closed under composition.
			
				Third, we need to show that the inverse of $Aut(G)$ is in $Aut(G)$. Let $f \in Aut(G)$, then we need to prove $f^{-1}$ in $Aut(G)$, $f^{-1}$ is an isomorphism from $G$ to $G$. Let $f^{-1} (x) = f^{-1}(y)$ such that $x,y \in G$. Since $f$ is bijective, we have:
				\begin{align*}
					f^{-1} (x) &= f^{-1}(y)\\
				f(f^{-1} (x)) &= f(f^{-1}(y))\\
					\epsilon(x) &= \epsilon(y)\\
					x&=y\\
				\end{align*}
			Hence $f^{-1}$ is injective.
			Now we need to show that for all $y \in G$, there exists $x \in G$ such that $f^{-1}(x) = y$. Since $f$ is surjective, there must exist $x \in G$ such that $f(y) =x$, implying that $y = f^{-1} x$. Hence, $f^{-1}$ is surjective and therefore bijective.
			
			To show that $f^{-1}$ is isomorphism to $G$, we also need to show that $f^{-1} (xy)= f^{-1}(x)f^{-1}(y).$ Let $a,b \in G$ such that $f^{-1} x= a $ and  $f^{-1}y =b$. $f^{-1}(x) f^{-1}(y)= ab$.Since $f$ is an isomorphism, we know that $f(ab) = f(a)f(b) = xy$. Then $f^{-1}(xy) = ab = f^{-1}(x) f^{-1}(x) $. Hence, the inverse of $f \in Aut(G)$ is in $G$.  Therefore, Aut(G) is a subgroup of $G$. 
			
			
			\end{proof}
		
		\item We'll prove some basic properties of order. Let $a, b, c\in G$. Show that
			\begin{enumerate}[label=(\alph*)]
			\item ord$(a) = $ ord$(bab^{-1})$
			\begin{proof}
				Let $ord(a) = x$ such that $x$ is the smallest positive integer that the equation holds.Then $a^x = e$. Need to show that $(bab^{-1})^x = e$.
				
				\begin{align*}
					(bab^{-1})^x &=  (bab^{-1})(bab^{-1})...(bab^{-1})\\
					&=bab^{-1}bab^{-1}...bab^{-1}\\
					& = ba(b^{-1}b)ab^{-1}...bab^{-1}\\
					& = ba^x b^{-1}\\
					& = beb^{-1}\\
					& = e
				\end{align*}
			
			Next, we need to show that $x$ is the smallest positive integer such that $(bab^{-1})^x = e$. Suppose there exists a positive integer $y$ such that $(bab^{-1})^y = e$ and $y < x$. Then, we can write $a^y = b^{-1}ba^y b^{-1}b = b^{-1}(bab^{-1})^y b = b^{-1}e b = e$, which contradicts the fact that $x$ is the smallest positive integer such that $a^x = e$. Therefore, $x$ is the smallest positive integer such that $(bab^{-1})^x = e$.
			Therefore, the order of $(bab^{-1})$ is $x$, which equals to the order of $a$.
			\end{proof}
			\item ord$(a^{-1}) = $ ord$(a)$
			\begin{proof}
					Let $ord(a) = x$ such that $x$ is the smallest positive integer that the equation holds.Then $a^x = e$. Need to show that $(a^{-1})^x = e$. According to the law of exponents, we know that $(a^{-1})^x  = (a^{x})^{-1} = e^{-1} =e$.
				
					Next, we need to show that $x$ is the smallest positive integer such that $(a^{-1})^x = e$. Suppose there exists a positive integer $y$ such that $(a^{-1})^y = e$ and $y < x$.  Then, we can write $a^y = ((a^{-1})^y)^{-1}= e^{-1} =e$, which contradicts the fact that $x$ is the smallest positive integer such that $a^x = e$. Therefore, $x$ is the smallest positive integer such that $(a^{-1})^x = e$. 
					Therefore, the order of $(a^{-1})$ is $x$, which equals to the order of $a$.
					
			\end{proof}
		\end{enumerate}
		\item Now show
		\begin{enumerate}[label=(\alph*)]
			\item ord$(ab) = $ ord$(ba)$
			\begin{proof}
			Let $ord(ab) = x$ such that $x$ is the smallest positive integer that the equation holds.Then $(ab)^x = e$. Need to show that $(ba)^x = e$. 
			
			\begin{align*}
				(ab)^x &= (ab)(ab)(ab)...(ab)\\
				&= a(ba)(ba)b...(ab)b\\
				&= a (ba)^{x-1} b\\
			\end{align*}
		
		Since $a (ba)^{x-1} b = (ab)^x  = e$. $(ba)^{x-1} $ must be $a^{-1}b^{-1}$, which equals to $(ba)^{-1}$. Since  $(ba)^{x-1}=  (ba)^{-1}$, we can conclude that $(ba)^x = (ba)^{-1} (ba)= e $.
		
			Next, we need to show that $x$ is the smallest positive integer such that $(ba)^x = e$. Suppose there exists a positive integer $y$ such that $(ba)^y = e$ and $y < x$.  Then, we can write $(ab)^y =a (ba)^{y-1} b = a (ba)^{y}(ba)^{-1} b  = a(ba)^{-1} b = a a^{-1} b^{-1} b = e$, which contradicts the fact that $x$ is the smallest positive integer such that $(ab)^x = e$. Therefore, $x$ is the smallest positive integer such that $(ba)^x = e$. 
		Therefore, the order of $ba$ is $x$, which equals to the order of $ab$.		
			\end{proof}
			\item ord$(abc) = $ ord$(cab) = $ ord$(bca)$
			
				Let $ord(abc) = x$ such that $x$ is the smallest positive integer that the equation holds.Then $(abc)^x = e$. Need to show that $(cab)^x = (bca)^x = e$. 
				
				\begin{align*}
					(abc)^x &= (abc)(abc)(abc)...(abc)\\
				    &= ab(cab)(cab)...(cab)c\\
				    & = ab(cab)^{x-1} c\\
				\end{align*}
			Since $ab(cab)^{x-1} c = 	(abc)^x = e$, $(cab)^{x-1}$ must equal to $b^{-1}a^{-1}c^{-1} = (cab)^{-1}$. Since $(cab)^{x-1} = (cab)^{-1}$, $(cab)^{x} =  (cab)^{-1} (cab) = e$. Similarly, we can get that $(bca)^{x} = e$ as well by realizing that $a(bca)^{n-1}bc =e$, $(bca)^{n-1} = (bca)^{-1} $ and $(bca)^{x} =e$.
					
			Next, we need to show that $x$ is the smallest positive integer such that $(cab)^x = e$. Suppose there exists a positive integer $y$ such that $(cab)^y = e$ and $y < x$.  Then, we can write $(abc)^y =ab (cab)^{y-1} c = ab(cab)^{y-1} c =ab(cab)^{y} (cab)^{-1} c = ab (cab)^{-1} c = e$, which contradicts the fact that $x$ is the smallest positive integer such that $(abc)^x = e$. Therefore, $x$ is the smallest positive integer such that $(cab)^x = e$. Similarly, $x$ is the smallest positive integer such that $(bca)^x = e$.
			
				Therefore, the order of $abc$ is $x$, which equals to the order of $cab$ and $bca$.
				
		\end{enumerate}
		
		\item Let $a\in G$, and of finite order. Prove that if $a$ is the $only$ element of order $k$ in $G$, then $a$ is in the center of $G$.
		
		\begin{proof}
			To prove that $a\in G$ is in the center of $G$, we need to show that for all $b\in G$,$ab = ba$.
			We proved in 3(a) that 	 ord$(a) = $ ord$(bab^{-1})$ for all $a,b \in G$. However, we know that $a$ is the only element of order $k$ in $G$, which implies that $a=bab^{-1}$. 
			\begin{align*}
				a&=bab^{-1}\\
				ab&=bab^{-1}b\\
				ab&=ba\\
			\end{align*}
			Therefore, $a$ is in the center of $G$.
		\end{proof}
		
	\end{enumerate}
\end{document}