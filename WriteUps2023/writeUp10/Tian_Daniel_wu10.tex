\documentclass[12pt]{article}
\usepackage{amsmath}
\usepackage{amssymb,amsfonts,latexsym,pgfplots,polynom,mathpazo,enumitem,textcase,bm,amsthm,fancyhdr}
\usepackage[utf8]{inputenc}
\usepackage[english]{babel}
\usepackage[margin=.9in, tmargin=1.5in, bmargin=1in]{geometry}
\usepackage{physics}
\usepackage{diffcoeff}
\usepackage{listings}
\usepackage{amsfonts} 

%pgfplots stuff
\pgfplotsset{width=10cm,compat=1.9}
\usepgfplotslibrary{external}
%\tikzexternalize 


%Theorems and stuff
\newtheorem{theorem}{Theorem}[section]
\newtheorem{corollary}{Corollary}[theorem]
\newtheorem{lemma}[theorem]{Lemma}
\newtheorem*{remark}{Remark}
\newtheorem{definition}{Definition}[section]

%Sets and stuff
\newcommand{\Z}{\mathbb{Z}}
\newcommand{\R}{\mathbb{R}}
\newcommand{\Q}{\mathbb{Q}}
\newcommand{\N}{\mathbb{N}}
\newcommand{\J}{\mathbb{J}}
\newcommand{\C}{\mathbb{C}}

\renewcommand\qedsymbol{$\blacksquare$}
\newenvironment{solution}
{\begin{proof}[Solution]\renewcommand\qedsymbol{$\square$}}
	{\end{proof}}

%Sequences and basic analysis
\newcommand{\seq}[1]{\{{#1}_n\}_{n=1}^\infty}
\newcommand{\seqk}[1]{\{{#1}_k\}_{k=1}^\infty}
\newcommand{\sseq}[1]{\{{#1}_{n_k}\}_{k=1}^\infty}
\newcommand{\script}[1]{\mathcal{#1}}
\newcommand{\Lim}[1]{\lim\limits_{{#1}\rightarrow\infty}}
\newcommand{\Limsup}[1]{\overline{\lim\limits_{{#1}\rightarrow\infty}}\textrm{ }}
\newcommand{\Liminf}[1]{\underline{\lim\limits_{{#1}\rightarrow\infty}}}
\newcommand{\re}{\textrm{Re}}
\newcommand{\im}{\textrm{Im}}

%making things bigger 
\newcommand{\Frac}[2]{\displaystyle\frac{#1}{#2}}
\newcommand{\Int}[2]{\displaystyle\int_{#1}^{#2}}
\newcommand{\Sum}[2]{\displaystyle\sum_{#1}^{#2}}
\newcommand{\Heq}{\overset{\mathrm{H}}{=}}
\newcommand{\dist}{\textrm{dist}}
\newcommand{\rpm}{\sbox0{$1$}\sbox2{$\scriptstyle\pm$}
	\raise\dimexpr(\ht0-\ht2)/2\relax\box2 }

%formatting
\newcommand\textlcsc[1]{\textsc{\MakeTextLowercase{#1}}}
\newcommand{\tab}{\hspace{10mm}}

%Vectors
\newcommand{\X}{\textbf{X}}
\newcommand{\Y}{\textbf{Y}}
\newcommand{\U}{\textbf{U}}
\newcommand{\vi}{\textbf{i}}
\newcommand{\vj}{\textbf{j}}
\newcommand{\vk}{\textbf{k}}
\newcommand{\vr}{\textbf{r}}
\newcommand{\vv}{\textbf{v}}
\newcommand{\vcu}{\textbf{u}}
\newcommand{\vca}{\textbf{a}}
\newcommand{\vcb}{\textbf{b}}
\newcommand{\vc}{\textbf{c}}
\newcommand{\la}{\langle}
\newcommand{\ra}{\rangle}

\newcommand{\pfpu}{\dfrac{\partial f}{\partial  u}}
\newcommand{\pfpv}{\dfrac{\partial f}{\partial  v}}
\newcommand{\pfpw}{\dfrac{\partial f}{\partial  w}}
\newcommand{\pfpx}{\dfrac{\partial f}{\partial  x}}
\newcommand{\pfpy}{\dfrac{\partial f}{\partial  y}}
\newcommand{\pupx}{\dfrac{\partial u}{\partial  x}}
\newcommand{\pupy}{\dfrac{\partial u}{\partial  y}}
\newcommand{\pupz}{\dfrac{\partial u}{\partial  z}}
\newcommand{\pvpx}{\dfrac{\partial v}{\partial  x}}
\newcommand{\pvpy}{\dfrac{\partial v}{\partial  y}}
\newcommand{\pvpz}{\dfrac{\partial v}{\partial  z}}
\newcommand{\pwpx}{\dfrac{\partial w}{\partial  x}}
\newcommand{\pwpy}{\dfrac{\partial w}{\partial  y}}
\newcommand{\pwpz}{\dfrac{\partial w}{\partial  z}}
\renewcommand\arraystretch{1.2}
%integrals
\def\upint{\mathchoice%
	{\mkern13mu\overline{\vphantom{\intop}\mkern7mu}\mkern-20mu}%
	{\mkern7mu\overline{\vphantom{\intop}\mkern7mu}\mkern-14mu}%
	{\mkern7mu\overline{\vphantom{\intop}\mkern7mu}\mkern-14mu}%
	{\mkern7mu\overline{\vphantom{\intop}\mkern7mu}\mkern-14mu}%
	\int}
\def\lowint{\mkern3mu\underline{\vphantom{\intop}\mkern7mu}\mkern-10mu\int}

%partials
\newcommand{\partd}[2]{\frac{\partial {#1}}{\partial {#2}}}
\newcommand{\partdd}[2]{\frac{\partial^2 {#1}}{\partial {#2}^2}}
\newcommand{\Partdd}[3]{\frac{\partial^2 {#1}}{\partial {#2}\partial{#3}}}
\newcommand{\Partddd}[4]{\frac{\partial^3 {#1}}{\partial {#2}\partial{#3}\partial{#4}}}
\newcommand{\Partdddd}[5]{\frac{\partial^4 {#1}}{\partial {#2}\partial{#3}\partial{#4}\partial{#5}}}

%words in math commands
\newcommand{\mathand}{\quad\textrm{and}\quad }
\newcommand{\st}{\textrm{ such that }}
\newcommand{\as}{\textrm{ as }}
\newcommand{\fs}{\textrm{ for some }}

%matrices
\newenvironment{amatrix}[1]{%
	\left[\begin{array}{@{}*{#1}{c}|c@{}}
	}{%
	\end{array}\right]
}



\pagestyle{fancy}
\fancyhf{}
\rhead{Daniel Tian\\ Advanced Topics in Math, Foil}   
\lhead{4/19/23}
\chead{\bf \large Write Up 10}
\cfoot{Page \thepage}

\begin{document}
	\begin{enumerate}
		
		\item If $G = \langle a\rangle$ and $b \in G$, then the order of $b$ is a factor of the order of $a$.
		\begin{proof}
			Let the order of $b$ be $l$. That is $b^l=e$. Since $b \in G$, $b$ is some $m$ power of $a$, then $b = a^m$. We can rewrite $b^l =e$ as $(a^m)^l = a^{ml}= e$. Hence, the order of b, $l$ is a factor of the order of $a$, $ml$.
			\end{proof}
		\item
		Let $G$ have order 4. Prove that either $G$ is cyclic, or every element of $G$ is its own inverse. Conclude that every group of order 4 is abelian.
		\begin{proof}
		Since $G$ is a finite group, and the possible order of its can only be 1, 2 and 4.	If $G$ have an element $a$ of order 1, $a^1 = e$, which is its own inverse. Then, if $G$ have an element $a$ of order 2,$a^2= e$. We can concludue that a is its own inverse because $a*a = e$. If $G$ have an element $a$ of order 4. Then $\langle a \rangle$ is a subgroup of $G$ and since $|a|=4$, we have $|\langle a \rangle|=4$. This means that $\langle a \rangle=G$ and so $G$ is cyclic. Then, Let $G$ have an element $b$ of order 2. Therefore,  either $G$ is cyclic, or every element of $G$ is its own inverse. 
		
		Cyclic group is abelian because it is all generated by the same generator. If every element is its own inverse. That implies $a,b \in G$, such that, $ab = (ab)^{-1} = b^{-1}a^{-1} =ba$. Therefore, every group of order 4 is abelian.
			
		\end{proof}
		\item
		Let $G$ be finite, and $H, K \leq G$. Suppose $H$ has index $p$ and $K$ has index $q$, where $p$ and $q$ are distinct primes. Prove that the index of $H \cap K$ is a multiple of $pq$.

		\begin{proof}
			Let the order of $G$ be $n$, order of $H$ be $l$ and the order of $K$ be $b$. Then, we know that the index of $H$, $p=\cfrac{n}{l}$ and the index of $K$, $q = \cfrac{n}{b}$. We know that $H \cap K$ is a subgroup of $H$ and of $K$. Then, by the Lagrange's theorem, we can conclude that $|H \cap K|$ is a factor of $|H|$ and of $|K|$. Let 	$|H \cap K| = z$, we can express $l = zi$ and $b=zj$. Hence, we can rewrite the $p = \cfrac{n}{zi}$ and $q = \cfrac{n}{zj}$. Or, $pi=qj = \cfrac{n}{z}$. We also know that $i$ must be an integer, then $p$ is a factor of $qj$. However, $q$ is a prime number, $p$ hence must be a factor of $j$, say $px =j$. Putting it in the equation, we will have $qj  = qpx= \cfrac{n}{z}$. Therefore, $qp = pq$ is a factor of $\cfrac{n}{z}$, which is the index of $H \cap K$. Then, $H \cap K$ is a muutiple of $pq$.
			
				
			\end{proof}
	
	\end{enumerate}

\end{document}