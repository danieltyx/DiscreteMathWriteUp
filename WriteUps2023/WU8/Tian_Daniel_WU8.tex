\documentclass[12pt]{article}
\usepackage{amsmath}
\usepackage{amssymb,amsfonts,latexsym,pgfplots,polynom,mathpazo,enumitem,textcase,bm,amsthm,fancyhdr}
\usepackage[utf8]{inputenc}
\usepackage[english]{babel}
\usepackage[margin=.9in, tmargin=1.5in, bmargin=1in]{geometry}
\usepackage{physics}
\usepackage{diffcoeff}

\DeclareUnicodeCharacter{2212}{-}

%pgfplots stuff
\pgfplotsset{width=10cm,compat=1.9}
\usepgfplotslibrary{external}
%\tikzexternalize 


%Theorems and stuff
\newtheorem{theorem}{Theorem}[section]
\newtheorem{corollary}{Corollary}[theorem]
\newtheorem{lemma}[theorem]{Lemma}
\newtheorem*{remark}{Remark}
\newtheorem{definition}{Definition}[section]

%Sets and stuff
\newcommand{\Z}{\mathbb{Z}}
\newcommand{\R}{\mathbb{R}}
\newcommand{\Q}{\mathbb{Q}}
\newcommand{\N}{\mathbb{N}}
\newcommand{\J}{\mathbb{J}}
\newcommand{\C}{\mathbb{C}}

\renewcommand\qedsymbol{$\blacksquare$}
\newenvironment{solution}
{\begin{proof}[Solution]\renewcommand\qedsymbol{$\square$}}
 {\end{proof}}

%Sequences and basic analysis
\newcommand{\seq}[1]{\{{#1}_n\}_{n=1}^\infty}
\newcommand{\seqk}[1]{\{{#1}_k\}_{k=1}^\infty}
\newcommand{\sseq}[1]{\{{#1}_{n_k}\}_{k=1}^\infty}
\newcommand{\script}[1]{\mathcal{#1}}
\newcommand{\Lim}[1]{\lim\limits_{{#1}\rightarrow\infty}}
\newcommand{\Limsup}[1]{\overline{\lim\limits_{{#1}\rightarrow\infty}}\textrm{ }}
\newcommand{\Liminf}[1]{\underline{\lim\limits_{{#1}\rightarrow\infty}}}
\newcommand{\re}{\textrm{Re}}
\newcommand{\im}{\textrm{Im}}

%making things bigger 
\newcommand{\Frac}[2]{\displaystyle\frac{#1}{#2}}
\newcommand{\Int}[2]{\displaystyle\int_{#1}^{#2}}
\newcommand{\Sum}[2]{\displaystyle\sum_{#1}^{#2}}
\newcommand{\Heq}{\overset{\mathrm{H}}{=}}
\newcommand{\dist}{\textrm{dist}}
\newcommand{\rpm}{\sbox0{$1$}\sbox2{$\scriptstyle\pm$}
 \raise\dimexpr(\ht0-\ht2)/2\relax\box2 }

%formatting
\newcommand\textlcsc[1]{\textsc{\MakeTextLowercase{#1}}}
\newcommand{\tab}{\hspace{10mm}}

%Vectors
\newcommand{\X}{\textbf{X}}
\newcommand{\Y}{\textbf{Y}}
\newcommand{\U}{\textbf{U}}
\newcommand{\vi}{\textbf{i}}
\newcommand{\vj}{\textbf{j}}
\newcommand{\vk}{\textbf{k}}
\newcommand{\vr}{\textbf{r}}
\newcommand{\vv}{\textbf{v}}
\newcommand{\vcu}{\textbf{u}}
\newcommand{\vca}{\textbf{a}}
\newcommand{\vcb}{\textbf{b}}
\newcommand{\vc}{\textbf{c}}
\newcommand{\la}{\langle}
\newcommand{\ra}{\rangle}

\newcommand{\pfpu}{\dfrac{\partial f}{\partial  u}}
\newcommand{\pfpv}{\dfrac{\partial f}{\partial  v}}
\newcommand{\pfpw}{\dfrac{\partial f}{\partial  w}}
\newcommand{\pfpx}{\dfrac{\partial f}{\partial  x}}
\newcommand{\pfpy}{\dfrac{\partial f}{\partial  y}}
\newcommand{\pupx}{\dfrac{\partial u}{\partial  x}}
\newcommand{\pupy}{\dfrac{\partial u}{\partial  y}}
\newcommand{\pupz}{\dfrac{\partial u}{\partial  z}}
\newcommand{\pvpx}{\dfrac{\partial v}{\partial  x}}
\newcommand{\pvpy}{\dfrac{\partial v}{\partial  y}}
\newcommand{\pvpz}{\dfrac{\partial v}{\partial  z}}
\newcommand{\pwpx}{\dfrac{\partial w}{\partial  x}}
\newcommand{\pwpy}{\dfrac{\partial w}{\partial  y}}
\newcommand{\pwpz}{\dfrac{\partial w}{\partial  z}}
\renewcommand\arraystretch{1.2}
%integrals
\def\upint{\mathchoice%
 {\mkern13mu\overline{\vphantom{\intop}\mkern7mu}\mkern-20mu}%
 {\mkern7mu\overline{\vphantom{\intop}\mkern7mu}\mkern-14mu}%
 {\mkern7mu\overline{\vphantom{\intop}\mkern7mu}\mkern-14mu}%
 {\mkern7mu\overline{\vphantom{\intop}\mkern7mu}\mkern-14mu}%
 \int}
\def\lowint{\mkern3mu\underline{\vphantom{\intop}\mkern7mu}\mkern-10mu\int}

%partials
\newcommand{\partd}[2]{\frac{\partial {#1}}{\partial {#2}}}
\newcommand{\partdd}[2]{\frac{\partial^2 {#1}}{\partial {#2}^2}}
\newcommand{\Partdd}[3]{\frac{\partial^2 {#1}}{\partial {#2}\partial{#3}}}
\newcommand{\Partddd}[4]{\frac{\partial^3 {#1}}{\partial {#2}\partial{#3}\partial{#4}}}
\newcommand{\Partdddd}[5]{\frac{\partial^4 {#1}}{\partial {#2}\partial{#3}\partial{#4}\partial{#5}}}

%words in math commands
\newcommand{\mathand}{\quad\textrm{and}\quad }
\newcommand{\st}{\textrm{ such that }}
\newcommand{\as}{\textrm{ as }}
\newcommand{\fs}{\textrm{ for some }}
\newcommand{\fsoc}{\textrm{ for the sake of contradiction}}


%matrices
\newenvironment{amatrix}[1]{%
 \left[\begin{array}{@{}*{#1}{c}|c@{}}
 }{%
 \end{array}\right]
}

%prove by contradiction
\newcommand{\con}{\Rightarrow \Leftarrow}


\pagestyle{fancy}
\fancyhf{}
\rhead{Daniel transposition\\ Advanced Topics in Math, Foil}   
\lhead{3/10/2023}
\chead{\bf \large Write Up 8}
\cfoot{Page \thepage}


\begin{document}
    \begin{enumerate}
 \item Let $G$ be a group, $a\in G$ and $f: G\to G$ is defined by $x\mapsto ax$. Determine if $f$ is injective, and if it is surjective. Provide proof or counterexample of your claims.

 \begin{proof}
    Let $x,y \in G$. Hence $f(x) = ax$ and $f(y) = ay$. Then, Let $f(x) = f(y)$ and we need to proof it implies $x=y$ for it to be injective.
    \begin{align*}
        f(x) &= f(y) \\
        ax &= ay \\
        a^{-1}ax & = a^{-1}ay\\ 
        x&= y\\
    \end{align*}
    Hence, $f$ is injective.\\
    Let $y \in G$, such that for all element $y$, there exist $x\in G$ such that $f(x) = y$.
    \begin{align*}
        y&= ax\\
        a^{-1}y &=  a^{-1}ax\\
        x &= a^{-1}y \\
    \end{align*}
    Since both $a^{-1}$ and $y$ are in $G$, $x$ must also be in $G$. Therefore, for all $y \in G$, there exist $f(a^{-1}y) = aa^{-1}y = y$, which indicates that $f$ is also surjective.

 \end{proof}
        \item Let $f: A\to B$ and $g: B\to A$. Suppose that $f(x) = y \text{ iff } g(y) = x$. Prove that $f$ is bijective (and that $g = f^{−1}$).
     \begin{proof}
        We need to prove $f$ is bijective that is $f$ is both injective and surjective.\\
        % Let $a,b \in A$ such that $f(a) = f(b)$ and let $f(a) = c$ and $f(b) = d$. Then, $g(f(a)) = g(c)$ and $g(f(b))= g(d)$. Having that $f(a) = f(b)$, we know that $c=d$ and $g(c) = g(d)$.
        % Since $f(x) = y$ iff $g(y) = x$ and $f(a) = c$ and $f(b) = d$, we can conclude that $a= g(c) = g(d) =b$ and $f$ is injective.\\
        % Let $y \in B$, and we need to prove that for all $y \in B$ there exists $x \in A$ such that $f(x) = y$ and hence $g(y) = x$s. 
        % $y= f(x) = f(g(y))$. 
        First, suppose that $f(x_1) = f(x_2)$ for some $x_1$, $x_2$ in $A$. We need to show that $x_1 = x_2$.
        Since $f(x_1) = f(x_2)$, we have $g(f(x_1)) = g(f(x_2))$. By the statement, it implies that $x_1 = x_2$. Therefore, $f$ is injective.\\

        Then, let $y \in B$, and we need to show that there exists an $x$ in $A$ such that $f(x) = y$.
        Since, $g$ is from $B$ to $A$, there is $x$ such that $g(y) = x$. Then, by the condition, we will have $f(x) = y$. Therefore, $f$ is surjective and hence bijective.\\
        
        We know that $f(x) = f(g(y)) = y$ and $g(f(x)) = g(y) = x$. If $x=y$, then $ f(g(y))  = g(f(x)) = y = x$, which implies that $g = f^{-1}$.
     \end{proof}
      
        
        \item Let $A = \{x \in \R \mid x \neq 0, 1\}$ and $G = \{\epsilon, f , g, h, j, k\}$, where
            \begin{align*}
               \epsilon = x \qquad &f = 1 − x \qquad g = \frac{1}{x}\\
               h = \frac{1}{1-x} \qquad &j = \frac{x-1}{x} \qquad d = \frac{x}{x-1}
            \end{align*}
        Show that $G \leq S_A$, and write the table out for $G$. ({\large H}INT: let the table do the heavy lifting for you.)
        
        \begin{proof}
            To show that $G \leq S_A$, we need to show that $G$ is a subgroup of $S_A$. Since $x \in \R$ and $x \neq 0,1$, all element in $G$ is in real number, hence in $A$.
            $S_A$ contains all permutations of $A$. Therefore, $G$ must be a nonempty subset of $S_A$.             

Then, we need to show that $G$ is closed under function composition operation. We can prove it by building the table.


$\epsilon \cdot \epsilon =\epsilon \cdot x = x = \epsilon$\\
$\epsilon \cdot f = \epsilon \cdot (1-x) = 1-x = f$\\
$\epsilon \cdot g =\epsilon \cdot \frac{1}{x} = \frac{1}{x} =g$\\
$\epsilon \cdot h =\epsilon \cdot \frac{1}{1-x} = \frac{1}{1-x} = h$\\
$\epsilon \cdot j =\epsilon \cdot \frac{x-1}{x} = \frac{x-1}{x} =j$\\
$\epsilon \cdot d =\epsilon \cdot \frac{x}{x-1} = \frac{x}{x-1} = d$ \\

$f \cdot \epsilon = f(x) = 1-x =f$ \\
$f \cdot f = f(f(x)) = f(1-x) = x = \epsilon$\\
$f \cdot g = f(g(x)) = f(\frac{1}{x}) = 1-\frac{1}{x} = \frac{x-1}{x} = j$\\
$f \cdot h = f(h(x)) = f(\frac{1}{1-x}) = \frac{x}{1-x} = d$\\
$f \cdot j = f(j(x)) = f(\frac{x-1}{x}) = \frac{1}{x} = g$\\
$f \cdot d = f(d(x)) = f(\frac{x}{x-1}) = \frac{x-1-x}{x-1} = -\frac{1}{x-1} =\frac{1}{1-x}=h$\\

$g \cdot \epsilon = g(x) = \frac{1}{x}$ = g\\
$g \cdot f = g(f(x)) = g(1-x) = \frac{1}{1-x} = h$\\
$g \cdot g = g(g(x)) = g(\frac{1}{x}) = x = \epsilon$\\
$g \cdot h = g(h(x)) = g(\frac{1}{1-x}) = 1-x = f$ \\
$g \cdot j = g(j(x)) = g(\frac{x-1}{x}) = \frac{x}{x-1}  =d$\\
$g \cdot d = g(d(x)) = g(\frac{x}{x-1}) = \frac{1}{\frac{x}{x-1}} = \frac{x-1}{x} = j$\\

$h \cdot \epsilon = h(x) = \frac{1}{1-x} = h$\\
$h \cdot f = h(f(x)) = h(1-x) = \frac{1}{1-(1-x)} = \frac{1}{x} = g$\\
$h \cdot g = h(g(x)) = h(\frac{1}{x}) = \frac{1}{1-\frac{1}{x}} = \frac{1}{\frac{x-1}{x}}=\frac{x}{x-1}=d$\\
$h \cdot h = h(h(x)) = h(\frac{1}{1-x}) = \frac{1}{1-\frac{1}{1-x}}= \frac{1}{\frac{x}{x-1}}= \frac{x-1}{x}=j$\\
$h \cdot j = h(j(x)) = h(\frac{x-1}{x}) = \frac{1}{1-\frac{x-1}{x}} = \frac{1}{\frac{1}{x}}=x=\epsilon$\\
$h \cdot d = h(d(x)) = h(\frac{x}{x-1}) = \frac{1}{1-\frac{x}{x-1}} = \frac{1}{\frac{1}{1-x}}=f$\\

$j \cdot \epsilon = j(x) = \frac{x-1}{x} = j$\\
$j \cdot f = j(f(x)) = j(1-x) = \frac{1-x-1}{1-x}=  \frac{x}{x-1} = d$\\
$j \cdot g = j(g(x)) = j(\frac{1}{x}) = \frac{\frac{1}{x}-1}{\frac{1}{x}} = 1-x = f$\\
$j \cdot h = j(h(x)) = j(\frac{1}{1-x}) = \frac{\frac{1}{1-x}-1}{\frac{1}{1-x}} = \frac{\frac{x}{1-x}}{\frac{1}{1-x}}=x = \epsilon$\\
$j \cdot j = j(j(x)) = j(\frac{x-1}{x}) = \frac{\frac{x-1}{x}-1}{\frac{x-1}{x}} = \frac{1}{1-x} =h$\\
$j \cdot d = j(d(x)) = j(\frac{x}{x-1}) = \frac{\frac{1}{x-1}}{\frac{x}{x-1}} = \frac{1}{x} =g$\\


$d \cdot \epsilon = d(x) = \frac{x}{x-1} = d$\\
$d \cdot f = d(f(x)) = d(1-x) = \frac{1-x}{1-x-1} = \frac{1-x}{-x} = \frac{x-1}{x} =j$\\
$d \cdot g = d(g(x)) = d(\frac{1}{x}) = \frac{\frac{1}{x}}{\frac{1}{x}-1} = \frac{1}{1-x} =h$\\
$d \cdot h = d(h(x)) = d(\frac{1}{1-x}) = \frac{\frac{1}{1-x}}{\frac{1}{1-x}-1} = \frac{1}{x} =g$\\
$d \cdot j = d(j(x)) = d(\frac{x-1}{x}) = \frac{\frac{x-1}{x}}{\frac{x-1}{x}-1} = \frac{x-1}{-1} = 1-x = f$\\
$d \cdot d = d(d(x))= d(\frac{x}{x-1}) = \frac{\frac{x}{x-1}}{\frac{x}{x-1}-1} = \frac{x}{1} = x =\epsilon$\\



$\epsilon \cdot \epsilon = \epsilon$ \\
$\epsilon \cdot f = f(\epsilon) = 1-\epsilon = 1-x = f$\\
$\epsilon \cdot g = g(\epsilon) = \frac{1}{\epsilon} = \frac{1}{x} = g$\\
$\epsilon \cdot h = h(\epsilon) = \frac{1}{1-\epsilon} = \frac{1}{1-x} = h$\\
$\epsilon \cdot j = j(\epsilon) = \frac{\epsilon - 1}{\epsilon} = \frac{x-1}{x} =j$\\
$\epsilon \cdot d = d(\epsilon) = \frac{\epsilon}{\epsilon-1} = \frac{x}{x-1} = d$\\ 


\begin{center}
    \begin{tabular}{c|cccccc}
    $\circ$ & $\epsilon$ & $f$ & $g$ & $h$ & $j$ & $d$ \\
    \hline
    $\epsilon$ & $\epsilon$  & $f$ & $g$ & $h$ & $j$ & $d$ \\
    $f$ & $f$ & $\epsilon$ & $j$ & $d$ & $g$ & $h$ \\
    $g$ & $g$ & $h$ & $\epsilon$ & $f$ & $d$ & $j$ \\
    $h$ & $h$ & $g$ & $d$ & $j$ & $\epsilon$ & $f$ \\
    $j$ & $j$ & $d$ & $f$ & $\epsilon$ & $h$  & $g$ \\
    $d$ & $d$ & $j$ & $h$ & $g$ & $f$ & $\epsilon$\\
    \end{tabular}
    \end{center}

    By exhaustion, we proved that the group is closed under the composite function(for all operations the result is unique and within the group). 
    Also, noticing that the inverse is all defined because all elements can composite with one other element such that their product is the identity.
    Therefore, $G$ is a subgroup of $S_A$.
\end{proof}
        \item Let $\alpha$ and $\beta$ be disjoint cycles:
        \[\alpha = (a_1\ a_2...a_s)\qquad \beta = (b_1\ b_2...b_r)\]
            \begin{enumerate}[label=(\alph*)]
                \item Prove that for $n \in \Z^+$, $(\alpha\beta)^n = \alpha^n\beta^n$.
                
                \begin{proof}
                    Since $\alpha$ and $\beta$ are disjoint hence commute. That is $\alpha\beta = \beta\alpha$.
                    \begin{align*}
                        \alpha^n\beta^n &= \alpha... \alpha\alpha\alpha\beta\beta\beta...\beta \\
                        &= \alpha... \alpha\alpha\beta\alpha\beta\beta...\beta \\
                        &= \alpha... \alpha\beta\alpha\beta\alpha\beta...\beta \\
                    \end{align*}
                    keep substituting $\alpha\beta$ with $\beta\alpha$ until the first two cycles are $\alpha \beta$ we will have:\\
                    \begin{align*}
                        &= \alpha\beta...\alpha\beta \\
                    \end{align*}
                    Noticing that the total number of cycle $\alpha$ and $\beta$ is fixed, $\alpha\beta$ will repeat $n$ times, hence $\alpha^n\beta^n =  (\alpha\beta)^n$ .
                   
                \end{proof}
                \item Find a transposition $\gamma$ such that $\alpha \beta \gamma$ is a cycle. Then show that $\alpha \gamma \beta$ and $\gamma \alpha \beta$ are cycles.
                \begin{solution}
                    Since $\alpha$ and $\beta$ are disjoint, $\gamma$ must carry some $b$ to some $a$. We can make it to carry the last element of $b$ 
                    to the first element of $a$. That is $\gamma = (a_1,b_r)$. We will have $(a_1\ a_2...a_s)(b_1\ b_2...b_r)(a_1,b_r)$
                    
                    
                    
                    
                    We will have $(a_1\ a_2...a_s)(a_1,b_r)(b_1\ b_2...b_r)$. 
                    Starting with $b_1$, it will be carried to $b_2$,
                    $b_2$ to $b_3$ ... until $b_{r-2}$ is carried to $b_{r-1}$. Then, $b_{r-1}$ will be carried to $b_r$ in the third cycle, to $a_1$ in the second cycle and to $a_2$ in the first cycle. 
                    Continuing with $a_2$, it will be carried to $a_3$,
                    $a_3$ to $a_4$ ... until $a_{s-1}$ is carried to $a_s$. Then, $a_s$ will be carried to $a_1$. Lastly, $a_1$ will be carried to $b_r$. The cycle will then be $(b_1b_2...b_{r-1}a_2a_3...a_sa_1b_r)$\\
               
               For $\alpha \gamma \beta$, we have $(a_1\ a_2...a_s)(a_1,b_r) $
               
               
                \end{solution}
            \end{enumerate}

        

        \item If $\pi$ is a permutation in $S_n$, and $\alpha$ is a cycle, we call $\pi \alpha \pi^{-1}$ a \textbf{conjugate} of $\alpha$. Let $\alpha = (a_1\ a_2...a_s)$ show that its conjugate $\pi \alpha \pi^{-1}$ is the cycle
        \[(\pi(a_1)\ \pi(a_2)...\pi(a_s)).\]

 \end{enumerate}
\end{document}